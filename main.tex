%%%%%%%%%%%%%%%%%
% This is an example CV created using altacv.cls (v1.1, 21 November 2016) written by
% LianTze Lim (liantze@gmail.com), based on the 
% Cv created by BusinessInsider at http://www.businessinsider.my/a-sample-resume-for-marissa-mayer-2016-7/?r=US&IR=T
% 
%% It may be distributed and/or modified under the
%% conditions of the LaTeX Project Public License, either version 1.3
%% of this license or (at your option) any later version.
%% The latest version of this license is in
%%    http://www.latex-project.org/lppl.txt
%% and version 1.3 or later is part of all distributions of LaTeX
%% version 2003/12/01 or later.
%%%%%%%%%%%%%%%%

%% If you want to use \orcid or the
%% academicons icons, add "academicons"
%% to the \documentclass options. 
%% Then compile with XeLaTeX or LuaLaTeX.
% \documentclass[10pt,a4paper,academicons]{altacv}
\documentclass[10pt,a4paper,ragged2e]{altacv}

%% AltaCV uses the fontawesome and academicon fonts
%% and packages. 
%% See texdoc.net/pkg/fontawecome and http://texdoc.net/pkg/academicons for full list of symbols.
%% When using the "academicons" option,
%% Compile with LuaLaTeX for best results. If you
%% want to use XeLaTeX, you may need to install
%% Academicons.ttf in your operating system's font %% folder.


% Change the page layout if you need to
\geometry{left=1cm,right=9cm,marginparwidth=6.8cm,marginparsep=1.2cm,top=1cm,bottom=1cm}

% Change the font if you want to.

% If using pdflatex:
\usepackage[utf8]{inputenc}
\usepackage[T1]{fontenc}
\usepackage[default]{lato}
\usepackage{hyperref}

% If using xelatex or lualatex:
% \setmainfont{Lato}

% Change the colours if you want to
%\definecolor{VividPurple}{HTML}{3E0097}
\definecolor{VividPurple}{HTML}{000497}
\definecolor{SlateGrey}{HTML}{2E2E2E}
\definecolor{LightGrey}{HTML}{666666}
\colorlet{heading}{VividPurple}
\colorlet{accent}{VividPurple}
\colorlet{emphasis}{SlateGrey}
\colorlet{body}{LightGrey}

% Change the bullets for itemize and rating marker
% for \cvskill if you want to
\renewcommand{\itemmarker}{{\small\textbullet}}
\renewcommand{\ratingmarker}{\faCircle}

%% sample.bib contains your publications
\addbibresource{sample.bib}

\begin{document}
\name{Humberto Sousa Martins}
\tagline{Machine Learning - Data Science - Python Development}
\photo{3cm}{foto_cortada}
\personalinfo{
  % Not all of these are required!
  % You can add your own with \printinfo{symbol}{detail}
\email{humbertosm.eng@gmail.com}
\phone{73 99830-1539} 

\linkedin{\href{https://www.linkedin.com/in/humbertosm-datascience/}{HumbertoSM-DataScience}} \hspace{0.2cm}
\github{\href{https://github.com/HumbertoSM-DataScience}{HumbertoSM-DataScience}} 
}

%% Make the header extend all the way to the right, if you want. Extend the right margin by 8cm (=6.8cm marginparwidth + 1.2cm marginparsep)
\begin{adjustwidth}{}{-8cm}
\makecvheader
\end{adjustwidth}

\addnextpagesidebar[-1ex]{page2sidebar}

%% Provide the file name containing the sidebar contents as an optional parameter to \cvsection.
%% You can always just use \marginpar{...} if you do
%% not need to align the top of the contents to any
%% \cvsection title in the "main" bar.

\cvsection[page1sidebar]{Experiência}

\cvevent{Estágio em Data Science na Secretaria de Saúde de Viçosa}{}{2021 - 2022}{Viçosa - MG}\begin{itemize}

\item Estágio em Data Science no setor de vacinação da Secretaria de Saúde de Viçosa pela empresa ToBlue, focado no uso de dados para organizar a vacinação contra o Coronavírus, comunicar resultados das ações e mitigar os efeitos da pandemia. \\
\item Resultados: Reorganização e automação de tratamento de dados, coleta de dados pessoais a nível municipal, gerador de relatórios automatizados para contabilidade de vacinas, elaboração de algoritmo para detecção de erros e fraudes, chatbot eficiente para dúvidas sobre a vacinação, dashboard interativo de acesso público no site da Prefeitura.\\
\item Desenvolvi: Comunicação com o poder público e público variado, organização por demandas, flexibilidade com urgências, proatividade e responsabilidade com dados sigilosos. \\
\item Ferramentas: Análise, manipulação e limpeza de dados sigilosos com Python e Pandas, plataformas online Google Forms, Sheets e Drive. Dashboards no DataStudio. Chatbots pelo PromptChat.\\


\end{itemize}

\divider

\cvevent{Gerente Júnior de Transportes e Financeiro em Inkiri Piracanga}{}{2014 - 2017}{Piracanga, Maraú - BA}
\begin{itemize}

\item Trabalhei como gerente júnior na área de transportes e posteriormente financeiro no hotel e centro holístico Inkiri Piracanga, em Maraú-BA.
\item Desenvolvi: Atendimento ao público em inglês e espanhol, comunicação em equipe multi-nacional, ambiente dinâmico e imprevisível, responsabilidade com acesso a recursos financeiros, liderança e treinamento de equipe.

\end{itemize}
\cvsection{Projetos e Prêmios}

\cvevent{Medalha de Honra ao Mérito - UFV}{}{2022}{Viçosa - MG}
\begin{itemize}
\item Medalha concedida por destaque acadêmico na graduação em Bacharelado em Física pela UFV
\end{itemize}

\cvevent{TCC: Rede U-Net para Otimização Topológica}{}{2020 - 2022}{Viçosa - MG}
\begin{itemize}
\item Projeto de pesquisa desenvolvido como TCC para o curso de Bacharelado em Física da UFV, em parceria com o Laboratório de Topologia da UNICAMP, visando usar as capacidades da arquitetura U-Net em segmentação semântica para melhorar a eficiência de algoritmos de otimização topológica. \\
\item Resultados: Modelo de rede funcional com acurácia binária de 95,2\% e melhora na eficiência do algoritmo em 6 ordens de grandeza, relatório cientifico aprovado por banca multidisciplinar. \\
\item Desenvolvi: Comunicação acadêmica multidisciplinar com outra instituição, leitura e interpretação de artigos científicos, documentação e versionamento de código, escrita de relatório. \\
\item Ferramentas: Análise e manipulação de dados em Python, bibliotecas Pandas, Numpy, MatplotLib, Keras/TensorFlow, SkLearn, armazenamento e execução na nuvem: Google Drive e Google Colab, relatório em LaTeX no OverLeaf   \\

\end{itemize}

\cvevent{Hackaton AibaLab: 1º lugar com Projeto SIIR}{}{2021}{Viçosa - MG}
\begin{itemize}

\item Participação e condecoração com o primeiro lugar do Hackaton AibaLab, focado em soluções tecnológicas sustentáveis para a agroindústria. 
\item Resultados: A equipe SIIR (Sistema Autônomo e Inteligente de Irrigação) desenvolveu um protótipo de sistema de irrigação baseado em Arduíno e utilizando previsões de AI para otimizar uso de recursos hídricos e elétricos. Além de prêmios em dinheiro e equipamentos, o projeto recebeu aceleração de 2 meses com especialistas da área e potenciais clientes.\\
\item Desenvolvi: Comunicação rápida em equipe, ambiente dinâmico com prazos curtos, papel de liderança, elaboração de pitchs concisos e habilidades de persuasão e argumentação.
\item Ferramentas: Desenvolvimento de layouts gráficos no Figma e Canva, organização de equipe no Trello.\\

\end{itemize}

\divider

\cvevent{Pesquisa no exterior: Física Nuclear na República Tcheca}{}{2019}{Brno - República Tcheca}
\begin{itemize}

\item Bolsa de estágio científico na Universidade Tecnológica de Brno, Rep. Tcheca concedida pela INCBAC, focado no em simulações e desenvolvimento experimental de reatores nucleares a base de sais de Tório.\\
\item Resultados: Relatório das atividades e experimentos desenvolvidos, resultados publicados no site da INCBAC.
\item Desenvolvi: Língua e cultura estrangeira, comunicação e escrita acadêmica em inglês.\\
\item Ferramentas: Métodos estatísticos de partículas com Deimos. Modelagem e simulação de reatores nucleares com técnica de Monte Carlo.


\end{itemize}

\cvsection{Outros Projetos}

\begin{itemize}
\item Criação de dashboards no DataStudio para empresa UpCycle, focada na implementação de pontos de coleta automatizados para reciclagem.\\
\item Apresentação em congresso SIA-UFV: Criação de video-aula publicada no YouTube sobre o tema Introdução a Redes Neurais Convolucionais para Problemas-Modelo, focado na apresentação de conceitos básicos de redes neurais convolucionais e suas aplicações. \\
\item Projetos computacionais em Python, focados em simulações de termodinâmica estatística, física de partículas e estatística. \\
\item Voluntário na Tetris - Empresa Júnior de Engenharia Civil pela UTFPR, execução de projetos arquitetônicos e estruturais acessíveis e de interesse social.
\end{itemize}



\clearpage


\end{document}
